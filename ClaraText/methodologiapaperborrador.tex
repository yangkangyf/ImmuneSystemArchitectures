1. DNA extraction:
On 10 June 2015, during an alien species focused survey of oyster beds in the marine lake Grevelingen, The Netherlands, a colony of \textit{D. vexillum} was collected with a mussel dredge (coordinates: N$51^\circ$ 45.073’, E$3^\circ$ 55.664’). Directly after collection, the colony was preserved on ethanol 96\% and stored in a refrigerator at 4 degrees Celsius (GiMaRIS collection number AG4844). On 14 December 2015 eight little pieces, with a diameter of about 4 mm each, were cut of this colony with a sterile scalpel for DNA-extraction. A Kingfisherflex robot was used to extract the DNA from these pieces with the Nucleomag Tissue kit from Macherey Nagel. To lyse the cells, 200$\mu$l T1 lysis buffer was added to the wells of a 96-well plate. Eight of these wells were used for the DNA extraction of Didemnum vexillum. After adding a small piece of tissue, 25$\mu$l of Proteinase K(20mg/ml) was added and incubated at $56^\circ$C overnight. After the cells were lysed, 225 $\mu$l of the sample was added to the MB2 plate containing 360 $\mu$l MB2 binding buffer (35‐55\% ethanol, 20‐40\% sodium perchlorate) and 25 $\mu$l Magnetic beads. The robot then mixed the mixture and transferred the DNA that was attached to the magnetic beads to a series of wash buffers (20‐30\% ethanol). The MB3 plate was filled with 600 $\mu$l MB3 wash buffer, the MB4 plate with 600 $\mu$l MB4 wash buffer, and the MB5 plate with 600 $\mu$l MB5 wash buffer. To release the DNA from the magnetic beads, the robot proceeded after the wash buffer to the MB6 plate with 150 $\mu$l MB6 elution buffer (5mM Tris/HCl, pH8.5). The DNA dilution was stored in the fridge at $4^\circ$C. With the Nanodrop ND1000 the quality and quantity was tested for each of the 8 DNA extractions. Based on these analyses two samples of 100 $\mu$l each were selected and send to PacBio for further analyses, i.e. samples Dvex2 (694.4 ng/$\mu$l, 260/280: 1.86, 260/230: 2.18) and Dvex3 (246.3 ng/$\mu$l, 260/280: 1.88, 260/230: 2.20).
2. RNA extraction:
Falta lo de Arjan

3. DNA sequence data generation:
Due to low throughput and the presence of partialy degradated gDNA on the samples we call for a SMRT library construction protocol without DNA shearing and BluePippin size selection cutoff of 7kb or not size selection in order to prepare  3 libraries of 20kb SMRTbell to be run on 11 SMRT cells to produce 12.173 Gbp. PacBio sequencing data was generated at University of Washington PacBio Sequencing Services in an instrument PacBio RSII under P6/C4 chemistry.   Illumina sequencing followed the procedure discribed in (Velandia et al) except that due to a potentially higher error rate in the PacBio data, all read with a PhRed value of $ge$10 were retained and procesed via kmer matching by BBDuk. Finally 10.5 Gpb of paired reads and 1.3Gbp single reads were used to correct PacBio reads in further steps. 
4. Sequence data processing:
In order to improve and complete the genome assemby an hybrid and non conservative de novo assembly approach was followed combining two protocols of the hierarchical genome assembly process (HGAP) of SMRT Analysis version 2.3 (Pacific Biosciences, USA) and another mapping protocol to correct PacBio reads by with Illumina sequencing data. First, a set of Consensus sequences for single molecules (CCS) were generated from raw PacBio data by using RS\_ReadsOfInsert  specifying a MinCompletePasses of 2 and Minerror $<$ 0.5\% . Besides, it was also generated a second  set of pre-assembled corrected reads by RS\_PreAssembler protocol considering reads using a minimun leggth of 250 and QV of 75\% with 24 alignments of candidates per chunk. Finally, Raw PacBio data without HPGA correction was also filtered by using \textbf{dextract} (reads of a size cutoff of 150 bp and QV of 87\%), then large-scale high-accuracy PacBio correction through iterative short read consensus of the Illumina proccesed sequencing was performed by proovread-2.13.13.
5. Genome assembly
Celera Assembler 8.3rc2 was used to assembly the three set of reads with error rates of 0.006 for OVL, CNS and CGW and 0.04 for the MAX error rate to produce Unitigs, Scafolds, Singletons and degenerated reads. Finally Redundants was used to identify unique reads to calculate the assembly deep. 
6.Assessment of genome assembly quality: 
Poner los nombres y parametros del ensamblaje del Transcriptoma con trinity version, luego la predicci'on con Augustus version  y the results  were also subjected to Benchmarking Universal Single-Copy Orthologs (BUSCO) v cual version ) with the ????? databases to evaluate the completeness of the genome. Among tantos grupos de  total BUSCO groups searched, cuantos grupos core genes were completed and partially identified, respectively, leading to a total of tantos porcetage 91.2\% BUSCO genes in the D. vexillum genome. 






