\documentclass[11pt]{article}

\usepackage[letterpaper, margin=2cm]{geometry}
\usepackage{graphicx}
\usepackage[english]{babel}
\usepackage[utf8]{inputenc}
\usepackage{hyperref}
\usepackage[square,numbers,sort&compress]{natbib}
\usepackage{listings}
\usepackage{rotating}
\usepackage{placeins} %Force biblio at final document
\usepackage{epic,eepic}
\usepackage{rotating}
\usepackage{float}
\usepackage{array}
\usepackage{afterpage}
\usepackage{amsmath,amsfonts,amssymb}
\usepackage{booktabs}
%\usepackage[T1]{fontenc}
%\usepackage{sectsty}
\usepackage{adjustbox}

%opening
\title{Supplemental information \#4: A new strategy to characterize the domain 
architecture structure of proteins of the innate inmune system in tunicate 
species}
\author{Cristian A. Velandia-Huerto*, Ernesto Parra, Federico D. 
Brown, Adriaan Gittenberger, \\ Peter F. Stadler and Clara I. 
Berm\'{u}dez-Santana}
\begin{document}

\maketitle

\section*{Didemnum vexillum re-annotation}
\subsection*{Annotation of coding regions with \texttt{Augustus}}
\subsection*{Mapping previous ncRNA annnotation on new assembly}
Previous ncRNA annotation was retrieved from Velandia-Huerto, \textit{et al} \cite{} 
in fasta format. All the contigs which have been reported an ncRNA have been obtained
from the old \textit{D.\ vexillum} genome 
version\footnote{\url{http://tunicata.bioinf.uni-leipzig.de/Download.html}}. This
multifasta file was mapped onto the new genome with \texttt{blast}:

\begin{lstlisting}[language=bash, breaklines=true]
blastall -p blastb -d <DB> -i <QUERY> -F F -e 10e-5 -m 8 -o <OUT>
\end{lstlisting}


According to the set of blast parameters the number of contigs were increasing
into the new genome. At the same time, if one contig reported more than one
candidates into the new genome, was choosen this/those one (s) that reported the 
highest bitscore. Having this previous information as an additional source of 
information in order to clean the true position of the annotated ncRNAs in the
new genomes. After mapping all the candidates with \texttt{blast}, the true locations
were obtained after applying those filters:
\begin{itemize}
 \item Identity have to be $\geq 85$\%.
 \item E-value $\leq 10^{-10}$.
 \item Relation of sizes between the homology region of the query
  ($r_h$) and their calculated size ($r_s$) have to be $\frac{r_h}{r_s} \geq 0.9$
\end{itemize}

From the $264$ reported loci of ncRNAs, XY were retrieved in the
new genome, \texttt{GFF} file is reported along with all the coding and non-coding
regions on Supplemental File X.
\end{document}
