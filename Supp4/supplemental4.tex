\documentclass[11pt]{article}

\usepackage[letterpaper, margin=2cm]{geometry}
\usepackage{graphicx}
\usepackage[english]{babel}
\usepackage[utf8]{inputenc}
\usepackage{hyperref}
\usepackage[square,numbers,sort&compress]{natbib}
\usepackage{listings}
\usepackage{rotating}
\usepackage{placeins} %Force biblio at final document
\usepackage{epic,eepic}
\usepackage{rotating}
\usepackage{float}
\usepackage{array}
\usepackage{afterpage}
\usepackage{amsmath,amsfonts,amssymb}
\usepackage{booktabs}
%\usepackage[T1]{fontenc}
\usepackage{color}
\newcommand{\TODO}[1]{\begingroup\color{red}#1\endgroup}
%\usepackage{sectsty}
\usepackage{adjustbox}

%opening
\title{Supplemental information \#4: A new strategy to characterize the domain 
architecture structure of proteins of the innate inmune system in tunicate 
species}
\author{Cristian A. Velandia-Huerto*, Ernesto Parra, Federico D. 
Brown, Adriaan Gittenberger, \\ Peter F. Stadler and Clara I. 
Berm\'{u}dez-Santana}
\begin{document}

\maketitle

\section*{Didemnum vexillum re-annotation}
\subsection*{Annotation of coding regions with \texttt{Augustus}}
\TODO{Ernesto and Clara are working on that}.
\subsection*{Mapping previous ncRNA annnotation on new assembly}
Previous ncRNA annotation was retrieved from Velandia-Huerto, \textit{et al} \cite{} 
in fasta format. All the contigs which have been reported an ncRNA have been obtained
from the first reported assembly of the \textit{D.\ vexillum} 
genome\footnote{\url{http://tunicata.bioinf.uni-leipzig.de/Download.html}}. This
multifasta file was mapped onto the new genome with \texttt{lastz}:

\begin{lstlisting}[language=bash, breaklines=true]
lastz_32 <NEW_GENOME>[multiple] <OLD_GENOME> --chain C=0 E=150 H=0 K=4500 L=3000 M=254 O=600 Q=human_chimp.v2.q T=2 Y=15000 --format=maf+
\end{lstlisting}

Aligment files were retrieved in \texttt{maf} format and were parsed
with \texttt{Bio::AlignIO} \texttt{Bioperl} library. The criteria to
obtain the best genome coordinates was choosen based on the relation
between the length of the mapped region into the new genome ($m$) and
the original size of the query contig in the old genome ($s$). The
relation was defined as $R = \frac{m}{s}$, and were defined as the best 
mapping candidates those ones reported $R = 1$, but in order to retrieve
the maximum number of mapping between the two genome versions, $R \geq 0.90$
was also considered.

From $247$ contigs, was possible to map $212$ in the raw results after the 
mapping stage with $lastz$. After considering the $R$ relation, those results
were parsed, resulting in: $64$ ($R = 1$), $35$ ($0.95 \leq R < 1 $), 
$39$ ($0.90 \leq R < 0.95$) and $32$ ($0.85 \leq R < 0.90$), in total
$170$ contigs that reported high score mapping into the new genome.

Best candidates was choosen based on the final aligment score. For those
contigs that reported 1:many relations, those set of positions in the new
assembly was also considered for the following analysis.

Sequences from ncRNAs was obtained and mapped against the new \textit{D.\
vexillum} assembly with \texttt{blast}, as follows:

\begin{lstlisting}[language=bash, breaklines=true]
blastall -p blastb -d <DB> -i <QUERY> -F F -e 10e-5 -m 8 -o <OUT>
\end{lstlisting}

According to the set of blast parameters the number of contigs were increasing
into the new genome. At the same time, if one contig reported more than one
candidates into the new genome, was choosen this/those one (s) that reported the 
highest bitscore. Having this previous information as an additional source of 
information in order to clean the true position of the annotated ncRNAs in the
new genomes. After mapping all the candidates with \texttt{blast}, the true locations
were obtained after applying those filters:
\begin{itemize}
 \item Identity have to be $\geq 85$\%.
 \item E-value $\leq 10^{-10}$.
 \item Relation of sizes between the homology region of the query
  ($r_h$) and their calculated size ($r_s$) have to be $\frac{r_h}{r_s} \geq 0.9$
\end{itemize}

An additional confirmation step was performed using the Covariance Models from
\texttt{RFAM}v.11 onto the retrieved fasta sequences, using \texttt{infernal} package:
\begin{lstlisting}[language=bash, breaklines=true]
cmsearch -g -Z <NT number (Mb)> --toponly <FASTA> <CM>
\end{lstlisting}

True candidates was obtained as reported in \cite{}. Following this
methodology was possible to obtain $67$ candidates that passed all the filters, 
and additional $15$ candidates were included after a manual curation, based
on their reported E-value and bitscore. Most of these, included candidates that
failed to pass all the applied filters, due they have been detected as 
truncated sequences.
So, in this group that was possible to mapped directly on the new 
\textit{D.\ vexillum} $77$ ncRNAs. \TODO{Please discuss about the
presence of the same candidate multiple times in the new genome}.
%In this case 77 was obtained, but 17 was discarted due low alignment scores.
%The complete Best group was composed by: 94 ncRNAs.

The other $170$ candidates did not reported a crossing with the old genome
assembly in the pairwise-aligment. For that reason, the obtained blast 
mapping allowed to retrieve candidate coordinates into the new genome
assembly. Fasta sequences were retrieved and next, evaluated by secondary
alignments with correspondent CMs. From those candidates, final GFF3 file
reported additional $86$ mapped ncRNAs into the new assembly.

From the $264$ reported loci of ncRNAs, $163$ were retrieved in the
new genome, \texttt{GFF} file is reported along with all the coding and non-coding
regions on Supplemental File X.

%%Candidates that failed blast mapping strategy
After mapping the complete set of reported sequences on the new genome assembly
some ncRNAs failed to be searched by blast. In this case, in order to
determine if this ncRNA family is present, homology searches using 
Hidden Markov Models (HMMs) was applied directly with the missing $51$ families.

\end{document}
